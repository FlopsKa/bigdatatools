\documentclass[a4paper,11pt]{scrartcl}
\usepackage[ngerman]{babel}
\usepackage[utf8]{inputenc}
\usepackage{amsmath}
\begin{document}

\title{Analyzing Big Data Streams}
\author{Florian Kalinke%
   \thanks{E-mail: \texttt{flops.ka@gmail.com}}}
\date{November 2017}
\maketitle

\begin{abstract}
   Das ist die Kurzfassung.
\end{abstract}

\tableofcontents

\section{Einleitung}
Der Begriff \textit{Big Data} ist eines der aktuellen Buzzwords der Informatik.
Unternehmen speichern ihre Daten ab und versuchen Erkenntnisse aus dem
Datenbestand abzuleiten. Abhängig von diesen Ergebnissen können strategische
Entscheidungen getroffen werden, um dem Unternehmen so einen wirtschaftlichen
Vorteil zu ermöglichen. Historisch gesehen ist dieses Vorgehen bereits
etabliert - die zu analysierende Datenmenge steigt allerdings stark an und
schafft so neue Herausforderungen bei der Analyse der Daten.

Obwohl nicht genau definiert ist, ab wann es sich bei der Datenverarbeitung um
\textit{Big Data} handelt, hat sich die folgende Definition durchgesetzt:

Datenmengen, die zu groß, zu komplex oder zu schnelllebig sind, um sie mit
traditionellen Methoden der Datenhaltung zu speichern und auszuwerten werden
als Big Data bezeichnet. Diese Einordnung geht davon aus, dass die Daten einem
oder mehreren der 3 „V“s entsprechen:
\begin{description}
  \item[Volume] Die Datenmengen sind im Tera-, Peta- oder Exabytebereich.
  \item[Velocity] Die Daten müssen in Echtzeit verarbeitet und analysiert
    werden.
  \item[Variety] Die Daten müssen keinem bestimmten Schema entsprechen, sie
    sind sowohl strukturiert, semistrukturiert als auch unstrukturiert.
\end{description}

Bei der Verarbeitung der Daten wird zwischen der \textit{Batch-} und
\textit{Stream-}Verarbeitung differenziert. Die Batch-Verarbeitung geht davon
aus, dass auf einer begrenzten Menge von Daten operiert wird. Das bedeutet,
dass die Größe der Daten bekannt und endlich ist. Die Datenmenge kann
„vollständig“ verarbeitet werden. Im Gegensatz dazu betrachtet die
Stream-Verarbeitung die Verarbeitung von Daten, die erst während eines
zeitlichen Verlaufs verfügbar werden.

Betrachtet man exemplarisch die Analyse der Besucherzahlen einer Webseite, so
lassen sich über Batch-Verarbeitung beispielsweise die Fragen beantworten „Wie
viele Personen haben die Webseite gestern aufgerufen? Wie viele vorgestern?
Wie viele in der vergangenen Woche?

Analog betrachtet die Stream-Verarbeitung die Fragestellungen: Wie viele
Besucher waren in der letzten Minute auf der Seite aktiv? Wie viele in den
letzten 10 Sekunden? Sind aktuell Besucher auf der Seite? Über
Stream-Verarbeitung können diese Daten in Echzeit analysiert und die Ergebnisse
betrachtet werden.

Weitere Einsatzgebiete sind zum Beispiel:
\begin{description}
  \item[Analyse von Sensordaten] Dies kann zum Beispiel durch die Sensorik im
    Auto illustriert werden: Die Werte verschiedener Sensoren zum Beispiel
    Beschleunigungssensor, Geschwindigkeitssensor und Abstandssensor, werden in
    Echtzeit zusammengeführt.
  \item[IoT Devices] Mit der zunehmenden Verbreitung von IoT Devices nimmt auch
    hier die eh schon sehr hohe Datenmenge zu: An eine zentrale Stelle werden
    verschiedene Events von unterschiedlichen Geräten gemeldet, zum Beispiel
    die Helligkeit des Raumes, ob sich eine Person in einem Raum befindet und
    die aktuelle Raumtemperatur.
  \item[Fraud Detection] In Echtzeit soll für Überweisungen geprüft werden, ob
    diese vom eigentlichen Bankkunden ausgeführt wurde, oder ob ein Betrüger
    diese im Namen des Kunden initiiert hat. Überweisungen, die der Kunde nicht
    selbst (oder unbewusst) in Auftrag gegeben hat, sollen aufgehalten werden.
  \item[Complex Event Processing] Beim Complex Event Processing wird in einem
    Datenstrom nach vorher festgelegten Mustern gesucht. Das bedeutet der
    Anwender legt Abfragen (ähnlich zu SQL) fest und der Stream wird
    kontinuierlich auf das Auftreten der definierten Muster geprüft.
  \item[High Speed Trading] Abhängig von Schwankungen der Aktien einer Börse
    sollen passende Wertpapiere gekauft oder verkauft werden, um den eigenen
    Gewinn zu maximieren. Im Optimalfall finden die Aktionen schneller statt,
    als die Konkurrenz diese ausführen kann.
  \item[Klickstreamanalyse] Der Bewegungspfad des Besuchers einer Webseite wird
    aufgezeichnet und analysiert. Abhängig vom Verhalten des Nutzers kann
    dieser zum Beispiel klassifiziert werden und es wird passende Werbung
    gezeigt.
\end{description}

Alle Einsatzgebiete haben gemeinsam, dass es sich um „Echtzeitsysteme“ handelt:
Aus den anfallenden Daten sollen möglichst schnell Informationen gewonnen
werden. Oft ist es sogar so, dass die Qualität der Informationen, die aus den
Daten gewonnen werden können, bei längerer Verarbeitungsdauer abnimmt.

Die vorliegende Arbeit gibt einen allgemeinen Einblick in die Verarbeitung von
Streams und zeigt im Anschluss detailliert die Möglichkeiten der Verarbeitung
von Streams mit dem Apache Storm Framework.

% TODO Kapitelbeschreibung



\section{Datenhaltung bei Big Data}
Die Verarbeitung von den in der Einleitung beschriebenen Datenmengen bringt
eine Menge an Herausforderungen mit sich: Ein einzelner Rechner ist zu schwach,
um die Menge an Daten a) abzuspeichern und b) Berechnungen auf diesen Daten
auszuführen. Aus diesem Grund wird für die Verarbeitung von Big Data im
Normalfall \textit{horizontal} skaliert. Das bedeutet, dass die Verarbeitung
statt auf einem extrem performanten Rechner auf viele Rechner verteilt wird.
Hier kommt sogenannte \textit{Commodity Hardware} zum Einsatz: Handelsübliche
Rechner, die im Verbund agieren und auf die anfallende Arbeit aufgeteilt wird.

Reduziert man Big Data auf die zu speichernde Datenmenge gibt es drei
Hauptgründe, die Last auf verschiedene Rechner zu verteilen:

\begin{description}
  \item[Skalierbarkeit] Die Daten können aufgrund des Speicherplatzes, der
    Schreibgeschwindigkeit oder der Lesegeschwindigkeit nicht auf einem
    einzelnen Rechner gespeichert werden.
  \item[Fehlertoleranz / Ausfallsicherheit] Das System soll nicht von einer
    einzelnen Maschine abhängig sein. Fällt ein Rechner aus, kann ein anderer
    die Arbeit übernehmen. Je mehr Rechner an einem System beteiligt sind,
    desto wahrscheinlicher ist es, dass einzelne Rechner oder einzelne
    Komponenten ausfallen.
  \item[Latenz] Sind die Daten auf einem Rechner, der sich geographisch nah am
    Ort des Zugriffs befindet, so müssen Netwerkpakete eine geringere Strecke
    zurücklegen und der Zugriff wird beschleunigt.
\end{description}

Um Daten verteilt vorzuhalten, gibt es zwei Ansätze:

\begin{description}
  \item[Replikation] Die gleichen Daten werden auf verschiedenen Knoten
    vorgehalten, sind also redundant gespeichert. Fällt ein Knoten aus, so
    können die Daten von einem anderen Knoten gelesen werden.
  \item[Partitionierung] Eine große Datenbank wird in Teilmengen zerlegt, die
    dann verschiedenen Knoten zugewiesen werden.
\end{description}

Die beiden Ansätze werden in der Praxis häufig kombiniert eingesetzt.

\section{Das CAP-Theorem}
Das CAP-Theorem oder \textit{Brewer's Theorem} wurde öffentlich erstmalig im
Jahr 2000 auf dem \textit{Symposium on Principles of Distributed Computing}
vorgestellt und im Jahr 2002 formal bewiesen. Das Theorem besagt, dass ein
verteiltes System nicht mehr als zwei der folgenden drei Garantien bieten kann:

\begin{description}
  \item[Konsistenz] Das Lesen eines Wertes liefert immer das Ergebnis des
    letzten Schreibens.
  \item[Verfügbarkeit] Das Lesen eines Wertes liefert immer ein Ergebnis,
    allerdings ohne die Garantie, dass der letzte Schreibvorgang berücksichtigt
    wurde.
  \item[Partitionstoleranz] Das System funktioniert unabhängig von der Zahl der
    verlorenen oder verzögerten Netzwerkpakete.
\end{description}

Zu beachten ist hier, dass nicht explizit zwei Garantien gewählt werden müssen
- gewählt werden muss zwischen Konsistenz und Verfügbarkeit, wenn tatsächlich
ein Netwerkfehler auftritt.

Graphisch wird das CAP-Theorem häufig als Pyramide dargestellt:

% TODO Graphik vom CAP-Theorem

\section{Lambda Architektur}
Die Lambda-Architektur wurde 2011 von Nathan Marz im Blogpost „How to beat the
CAP-Theorem“ vorgestellt und beruht auf der Beobachtung, dass sich die
Komplexität in verteilten Systemen aus dem veränderlichen Zustand in
Datenbanken („mutable state“) und der Nutzung von inkrementellen Algorithmen
zur Manipulation dieses Zustands ergibt.

Der Autor wirft die Frage auf, wie ein System aussehen würde, dass als
Kernkomponente eine Append-Only-Datenbank verwendet - das heißt der Zustand
eines Datums in der Datenbank kann nicht verändert werden. Berechnungen werden
auf sämtlichen vorhandenen Daten ausgeführt und die Nutzung von inkrementellen
Algorithmen entfällt.

$\text{Query} = \text{Function(All Data)}$

Die Abfrage wird als Batch-Verarbeitung ausgeführt und das Ergebnis
abgespeichert. Eine Berechnung auf sämtlichen vorhandenen Daten kann sehr lange
dauern, dass heißt es gibt keine aktuelle Sicht auf die Daten.

Das Problem wird dadurch gelöst, dass historische Daten über einen Batch-Job
verarbeitet werden. Neu anfallende Daten verarbeitet der sogenannte Speed- bzw.
Realtime Layer und dieser füllt so die entstehende Lücke auf. Kombiniert werden
die beiden Sichten auf die Daten vom \textit{Serving Layer}.

\subsection{Batch Layer}
Der Batch-Layer geht davon aus, dass es ok ist, wenn es keine aktuelle Sicht
auf die Daten gibt und die Berechnungen dadurch entsprechend einfach werden.
Das heißt der Batch Job läuft, führt seine Berechnungen aus und schreibt das
Ergebnis in eine Datenbank. Dabei werden schon in der Datenbank bestehende
Daten vollständig ersetzt. Im Anschluss wird die Neuberechnung angestartet, die
die in der Zwischenzeit angefallenen Daten mitbetrachtet. Ist diese
abgeschlossen ersetzt das Ergebnis wieder das vorherige usw.

\subsection{Speed / Realtime Layer}
Der Speed-Layer füllt die Versorgungslücke, die entsteht, während die
Berechnungen des Batch-Layers laufen. Das bedeutet der veränderbare Zusatand,
der aus der Batch-Verarbeitung heraus gehalten wurde und die damit
einhergehenden inkrementellen Algorithmen befinden sich jetzt im Speed-Layer
und führen hier die Berechnungen in Echtzeit aus.

Die Verschiebung der Komplexität in den Speed-Layer bringt einige Vorteile mit
sich: Die Daten, die vom Speed-Layer berechnet werden, sind nur so lange
gültig, wie der Batch-Layer für seine Berechnungen braucht. Das bedeutet ein
Fehler bei der Datenanalyse kann den Betriebsfluss zwar stören - verschwindet
aber sobald das Batch-System die Daten in der Datenbank unter Einbeziehung des
neuen Wertes schreibt.

An dieser Stelle ist das System auch Fehlern des Entwicklers über toleranter -
es kann a) ein Fehler bei der Entwicklung der Verarabeitungslogik im
Batch-Layer programmiert werden, oder b) bei der Entwicklung des Speed-Layers.

Tritt der Fehler im Batch-Layer auf, so zieht sich der Fehler durch das gesamt
System und verfälscht die Auswertungen. Da jede Neuberechnung allerdings auf
sämtlichen historischen Daten basiert (Append-only) kann der Fehler vom
Entwickler korrigiert werden und die Datenbasis ist nach der Ausbringung der
Fehlerkorrektur und der nächsten Neuberechnung wieder korrekt. Zukünftige
Berechnungen des Speed-Layers und Abfragen auf den Datenbestand nutzen die
korrigiert Version, das heißt es gibt nur temporär eine verfälschte Datenbasis.

Beim Auftreten eines Fehlers im Speed-Layer, so ist wie oben bereits
angesprochen, nicht die gesamte Datenbasis vom Fehler beeinflusst, sondern nur
der Zeitraum, seit die letzte Komplettberechnung des Batch-Layers abgeschlossen
wurde. Davon ausgehend, dass die Berechnungen auf den Daten im Batch-Layer
korrekt sind, sind somit die historischen Daten nicht vom Fehler beeinflusst.

\subsection{Serving Layer}
Die eigentliche Komplexität der Lambda-Architektur verbirgt sich hinter dem
\textit{Serving Layer}. Dieser wird im ursprünglichen Artikel von Nathan Marz
nicht beschrieben und wurde erstmals XXXX genannt. 

Der Client des verteilten Systems soll im Optimalfall von der Aufteilung der
Berechnungen in den Batch-Layer und den Speed-Layer nichts mitbekommen. Das
heißt er muss eine übergreifende Sicht über beide zum Einsatz kommende
Datenbanken bekommen und Abfragen müssen die Daten aus beiden Datenbanken
zusammenfassen.

Die Aufgabe des Serving-Layers ist es, dem Client diese Sicht zur Verfügung zu
stellen und die Komplexität der Berechnungen im Hintergrund zu verbergen.

\section{Hadoop Kernkomponenten}
Das Hadoop Projekt bietet mittlerweile ein ganzes „Ökosystem“ an Anwendungen
für die Datenhaltung und die Datenanalyse. Ursprünglich wurden unter dem
Hadoop-Begriff die verteilte Speicherschicht HDFS und die verteilte
Rechenschicht MapReduce zusammengefasst. Diese bilden noch immer die Basis des
Anwendungsstacks und die strukturelle Umsetzung der beiden Komponenten wird im
Folgenden aufgezeigt.

\subsection{Hadoop Filesystem}

Das Hadoop Filesystem oder \textit{HDFS} (Hadoop Distributed Filesystem)
betrachtet allen im Cluster verfügbaren Speicherplatz als ein großes logisches
Volumen. Die Anwendung schreibt Daten nicht direkt auf einen Rechner, sondern
nutzt das von HDFS bereitgestellte Interface zur Ablage von Dateien. Auf
welchem Rechner eine Datei konkret abgelegt wird, entscheidet das verteilte
Dateisystem.

Die Daten werden dabei abhängig von der Konfiguration nicht nur auf einem
Knoten abgelegt, sondern redundant auf mehreren Knoten vorgehalten. Fällt ein
Knoten aus, so kann die Datei von einem anderen Knoten gelesen werden.

% TODO Standardkonfiguration 3-Fach -  für Schutz vor Server, Rack,
% Netzausfällen, Lesefehlern.

Um dieses Prinzip umzusetzen nutzt HDFS das Master-Slave-Prinzip: Der Master im
Cluster speichert, welche Daten auf welchen Slaves vorgehalten werden. Soll auf
eine Datei zugegriffen werden, geht zuerst eine Anfrage an den Master, um
herauszufinden, auf welchem Knoten sich die Datei aktuell befindet. Im
Anschluss wird die Datei direkt von dort abgerufen.

Ist ein Knoten für den Master nicht mehr erreichbar, so kümmert sich dieser
auch darum, dass nun nicht mehr ausreichend replizierte Daten wieder auf andere
Knoten gespiegelt werden. Da der Master in dieser Konfiguration einen
\textit{single-point-of-failure} darstellt bietet HDFS zusätzlich eine
High-Availability Konfiguration, bei der $n$ Master ($n \geq 2$) vorgehalten
werden. Beim Ausfall des aktuell aktiven Masters kann auf einen anderen Master
umgeschwenkt werden.

% TODO Namenode / Datanode

\subsection{MapReduce Framework}
Das MapReduce Framework bietet dem Entwickler eine klar definierte
Schnittstelle für das verteilte Rechnen. Die auf dem MapReduce Framework
aufsetzenden Algorithmen laufen in 3 Phasen ab. Diese sind schematisch in der
folgenden Abbildung dargestellt.

\begin{enumerate}
  \item Map-Phase
  \item Group- / Shuffle-Phase
  \item Reduce-Phase
\end{enumerate}

Die Eingabe des Algorithmus besteht aus Schlüssel- und Werte-Paaren
$(K_1,V_1)$. Der Map-Schritt übersetzt diese in eine Menge von neuen Schlüssel-
und Werte-Paaren $(K_2,V_2)$. Der zweite Schritt, der Group- oder
Shuffle-Schritt, fasst Paare mit dem gleichen Schlüssel zu $(K_2,
\{V_2,V_2,\ldots,V_2\})$ zusammen. Im letzten Schritt, dem Reduce-Schritt,
werden die verschiedenen Werte kombiniert. Daraus ergibt sich $(K_3,V_3)$.

Die Vorteile dieser Aufteilung sind die Möglichkeit der automatischen
Parallelisierung, eine hohe Fehlertoleranz und ein einfaches Verteilen der
verschiedenen Rechenschritte. Das Map-Reduce-Paradigma wird für die
Batch-Verarbeitung genutzt.

\section{Verarbeitung von Streams}
Die Verarbeitung von Streams geht davon aus, dass neu anfallende Daten in
Echtzeit analysiert und bewertet werden müssen. Dies können zum Beispiel
Sensordaten, Klick-Daten einer Webseite, Transaktionen eines Banksystems oder
Events in einem Rechenzentrum sein. Statt die Daten, wie beim klassischen
Data-Warehouse, erst zu indizieren und danach Abfragen auf den gespeicherten
Daten auszuführen, werden die Daten im laufenden Betrieb mit statistischen und
mathematischen Methoden analysiert.

Der Betrachter der Ergebnisse erhält so in Echtzeit Rückmeldung über ein
laufendes System und kann aus den ihm so zur Verfügung gestellten Informationen
Handlungen ableiten.

% TODO Wo kommen diese Informationen her? Message Broker: AMQP/JMS-Style Broker
% vs. Log-Based-Message Broker

Zur Analyse von Datenströmen in Echtzeit gibt es eine Vielzahl an Frameworks,
im Folgenden werden die OpenSource-Frameworks Apache Storm, Apache
Spark-Streaming und Apache Kafka kurz vorgestellt und miteinander verglichen.

% TODO Wieso diese Frameworks? => Open Source, können problemlos verprobt
% werden

% TODO Windowing Methoden
% TODO Avro / Serialisierung von Nachrichten

\subsection{Apache Storm}
\subsection{Apache Spark-Streaming}
\subsection{Apache Kafka}

\section{Echtzeitanalyse von Tastatureingaben}

% TODO Architekturschaubild
% TODO Zusammenarbeit jQuery, rest-Framework, Kafka, Storm

\section{Zusammenfassung / Fazit}


\end{document}
