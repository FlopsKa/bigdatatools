\documentclass[a4paper,11pt]{scrartcl}
\usepackage[ngerman]{babel}
\usepackage[utf8]{inputenc}
\usepackage{tikz}
\usepackage{amsmath}
\usepackage{listings}
\usetikzlibrary{shapes,decorations.pathreplacing,angles,quotes}

\definecolor{dkgreen}{rgb}{0,0.6,0}
\definecolor{gray}{rgb}{0.5,0.5,0.5}
\definecolor{mauve}{rgb}{0.58,0,0.82}

\lstset{frame=tb,
  language=Java,
  aboveskip=3mm,
  belowskip=3mm,
  showstringspaces=false,
  columns=flexible,
  basicstyle={\small\ttfamily},
  numbers=none,
  numberstyle=\tiny\color{gray},
  keywordstyle=\color{blue},
  commentstyle=\color{dkgreen},
  stringstyle=\color{mauve},
  breaklines=true,
  breakatwhitespace=true,
  tabsize=3
}

\begin{document}

\title{Analyzing Big Data Streams}
\author{Florian Kalinke%
  \thanks{E-mail: \texttt{flops.ka@gmail.com}}}
  \date{November 2017}
  \maketitle

  \begin{abstract}
    Das ist die Kurzfassung.
  \end{abstract}

  \tableofcontents

  \section{Einleitung}
  Der Begriff \textit{Big Data} ist eines der aktuellen Buzzwords der Informatik.
  Unternehmen speichern ihre Daten ab und versuchen Erkenntnisse aus dem
  Datenbestand abzuleiten. Abhängig von diesen Ergebnissen können strategische
  Entscheidungen getroffen werden, um dem Unternehmen so einen wirtschaftlichen
  Vorteil zu ermöglichen. Historisch gesehen ist dieses Vorgehen bereits
  etabliert - die zu analysierende Datenmenge steigt allerdings stark an und
  schafft so neue Herausforderungen bei der Analyse der Daten.

  Obwohl nicht genau definiert ist, ab wann es sich bei der Datenverarbeitung um
  \textit{Big Data} handelt, hat sich die folgende Definition durchgesetzt:

  Datenmengen, die zu groß, zu komplex oder zu schnelllebig sind, um sie mit
  traditionellen Methoden der Datenhaltung zu speichern und auszuwerten werden
  als Big Data bezeichnet. Diese Einordnung geht davon aus, dass die Daten einem
  oder mehreren der 3 „V“s entsprechen:
  \begin{description}
    \item[Volume] Die Datenmengen sind im Tera-, Peta- oder Exabytebereich.
    \item[Velocity] Die Daten müssen in Echtzeit verarbeitet und analysiert
      werden.
    \item[Variety] Die Daten müssen keinem bestimmten Schema entsprechen, sie
      sind sowohl strukturiert, semistrukturiert als auch unstrukturiert.
  \end{description}

  Bei der Verarbeitung der Daten wird zwischen der \textit{Batch-} und
  \textit{Stream-}Verarbeitung differenziert. Die Batch-Verarbeitung geht davon
  aus, dass auf einer begrenzten Menge von Daten operiert wird. Das bedeutet,
  dass die Größe der Daten bekannt und endlich ist. Die Datenmenge kann
  „vollständig“ verarbeitet werden. Im Gegensatz dazu betrachtet die
  Stream-Verarbeitung die Verarbeitung von Daten, die erst während eines
  zeitlichen Verlaufs verfügbar werden.

  Betrachtet man exemplarisch die Analyse der Besucherzahlen einer Webseite, so
  lassen sich über Batch-Verarbeitung beispielsweise die Fragen beantworten „Wie
  viele Personen haben die Webseite gestern aufgerufen? Wie viele vorgestern?
  Wie viele in der vergangenen Woche?

  Analog betrachtet die Stream-Verarbeitung die Fragestellungen: Wie viele
  Besucher waren in der letzten Minute auf der Seite aktiv? Wie viele in den
  letzten 10 Sekunden? Sind aktuell Besucher auf der Seite? Über
  Stream-Verarbeitung können diese Daten in Echzeit analysiert und die Ergebnisse
  betrachtet werden.

  Weitere Einsatzgebiete sind zum Beispiel:
  \begin{description}
    \item[Analyse von Sensordaten] Dies kann zum Beispiel durch die Sensorik im
      Auto illustriert werden: Die Werte verschiedener Sensoren zum Beispiel
      Beschleunigungssensor, Geschwindigkeitssensor und Abstandssensor, werden in
      Echtzeit zusammengeführt.
    \item[IoT Devices] Mit der zunehmenden Verbreitung von IoT Devices nimmt auch
      hier die eh schon sehr hohe Datenmenge zu: An eine zentrale Stelle werden
      verschiedene Events von unterschiedlichen Geräten gemeldet, zum Beispiel
      die Helligkeit des Raumes, ob sich eine Person in einem Raum befindet und
      die aktuelle Raumtemperatur.
    \item[Fraud Detection] In Echtzeit soll für Überweisungen geprüft werden, ob
      diese vom eigentlichen Bankkunden ausgeführt wurde, oder ob ein Betrüger
      diese im Namen des Kunden initiiert hat. Überweisungen, die der Kunde nicht
      selbst (oder unbewusst) in Auftrag gegeben hat, sollen aufgehalten werden.
    \item[Complex Event Processing] Beim Complex Event Processing wird in einem
      Datenstrom nach vorher festgelegten Mustern gesucht. Das bedeutet der
      Anwender legt Abfragen (ähnlich zu SQL) fest und der Stream wird
      kontinuierlich auf das Auftreten der definierten Muster geprüft.
    \item[High Speed Trading] Abhängig von Schwankungen der Aktien einer Börse
      sollen passende Wertpapiere gekauft oder verkauft werden, um den eigenen
      Gewinn zu maximieren. Im Optimalfall finden die Aktionen schneller statt,
      als die Konkurrenz diese ausführen kann.
    \item[Klickstreamanalyse] Der Bewegungspfad des Besuchers einer Webseite wird
      aufgezeichnet und analysiert. Abhängig vom Verhalten des Nutzers kann
      dieser zum Beispiel klassifiziert werden und es wird passende Werbung
      gezeigt.
  \end{description}

  Alle Einsatzgebiete haben gemeinsam, dass es sich um „Echtzeitsysteme“ handelt:
  Aus den anfallenden Daten sollen möglichst schnell Informationen gewonnen
  werden. Oft ist es sogar so, dass die Qualität der Informationen, die aus den
  Daten gewonnen werden können, bei längerer Verarbeitungsdauer abnimmt.

  Die vorliegende Arbeit gibt einen allgemeinen Einblick in die Verarbeitung von
  Streams und zeigt im Anschluss detailliert die Möglichkeiten der Verarbeitung
  von Streams mit dem Apache Storm Framework.

% TODO Kapitelbeschreibung

  \section{Datenhaltung bei Big Data}
  Die Verarbeitung von den in der Einleitung beschriebenen Datenmengen bringt
  eine Menge an Herausforderungen mit sich: Ein einzelner Rechner ist zu schwach,
  um die Menge an Daten a) abzuspeichern und b) Berechnungen auf diesen Daten
  auszuführen. Aus diesem Grund wird für die Verarbeitung von Big Data im
  Normalfall \textit{horizontal} skaliert. Das bedeutet, dass die Verarbeitung
  statt auf einem extrem performanten Rechner auf viele Rechner verteilt wird.
  Hier kommt sogenannte \textit{Commodity Hardware} zum Einsatz: Handelsübliche
  Rechner, die im Verbund agieren und auf die anfallende Arbeit aufgeteilt wird.

  Reduziert man Big Data auf die zu speichernde Datenmenge gibt es drei
  Hauptgründe, die Last auf verschiedene Rechner zu verteilen:

  \begin{description}
    \item[Skalierbarkeit] Die Daten können aufgrund des Speicherplatzes, der
      Schreibgeschwindigkeit oder der Lesegeschwindigkeit nicht auf einem
      einzelnen Rechner gespeichert werden.
    \item[Fehlertoleranz / Ausfallsicherheit] Das System soll nicht von einer
      einzelnen Maschine abhängig sein. Fällt ein Rechner aus, kann ein anderer
      die Arbeit übernehmen. Je mehr Rechner an einem System beteiligt sind,
      desto wahrscheinlicher ist es, dass einzelne Rechner oder einzelne
      Komponenten ausfallen.
    \item[Latenz] Sind die Daten auf einem Rechner, der sich geographisch nah am
      Ort des Zugriffs befindet, so müssen Netwerkpakete eine geringere Strecke
      zurücklegen und der Zugriff wird beschleunigt.
  \end{description}

  Um Daten verteilt vorzuhalten, gibt es zwei Ansätze:

  \begin{description}
    \item[Replikation] Die gleichen Daten werden auf verschiedenen Knoten
      vorgehalten, sind also redundant gespeichert. Fällt ein Knoten aus, so
      können die Daten von einem anderen Knoten gelesen werden.
    \item[Partitionierung] Eine große Datenbank wird in Teilmengen zerlegt, die
      dann verschiedenen Knoten zugewiesen werden.
  \end{description}

  Die beiden Ansätze werden in der Praxis häufig kombiniert eingesetzt.

  \section{Das CAP-Theorem}
  Das CAP-Theorem oder \textit{Brewer's Theorem} wurde öffentlich erstmalig im
  Jahr 2000 auf dem \textit{Symposium on Principles of Distributed Computing}
  vorgestellt und im Jahr 2002 formal bewiesen. Das Theorem besagt, dass ein
  verteiltes System nicht mehr als zwei der folgenden drei Garantien bieten kann:

  \begin{description}
    \item[Konsistenz] Das Lesen eines Wertes liefert immer das Ergebnis des
      letzten Schreibens.
    \item[Verfügbarkeit] Das Lesen eines Wertes liefert immer ein Ergebnis,
      allerdings ohne die Garantie, dass der letzte Schreibvorgang berücksichtigt
      wurde.
    \item[Partitionstoleranz] Das System funktioniert unabhängig von der Zahl der
      verlorenen oder verzögerten Netzwerkpakete.
  \end{description}

  Zu beachten ist hier, dass nicht explizit zwei Garantien gewählt werden müssen
  - gewählt werden muss zwischen Konsistenz und Verfügbarkeit, wenn tatsächlich
  ein Netwerkfehler auftritt.

  Graphisch wird das CAP-Theorem häufig als Pyramide dargestellt:

% TODO Graphik vom CAP-Theorem

  \section{Lambda Architektur}
  Die Lambda-Architektur wurde 2011 von Nathan Marz im Blogpost „How
  to beat the CAP-Theorem“ vorgestellt und beruht auf der Beobachtung,
  dass sich die Komplexität in verteilten Systemen aus dem
  veränderlichen Zustand in Datenbanken („mutable state“) und der
  Nutzung von inkrementellen Algorithmen zur Manipulation dieses
  Zustands ergibt.  Der Name leitet sich aus dem Lambda-Kalkül als
  Grundlage der funktionalen Programmierung ab: Berechnungen im
  Batch-Layer werden nur auf unveränderlichen Daten ausgeführt.

  Der Autor wirft die Frage auf, wie ein System aussehen würde, dass als
  Kernkomponente eine Append-Only-Datenbank verwendet - das heißt der Zustand
  eines Datums in der Datenbank kann nicht verändert werden. Berechnungen werden
  auf sämtlichen vorhandenen Daten ausgeführt und die Nutzung von inkrementellen
  Algorithmen entfällt.

  $\text{Query} = \text{Function(All Data)}$

  Die Abfrage wird als Batch-Verarbeitung ausgeführt und das Ergebnis
  abgespeichert. Eine Berechnung auf sämtlichen vorhandenen Daten kann sehr lange
  dauern, dass heißt es gibt keine aktuelle Sicht auf die Daten.

  Das Problem wird dadurch gelöst, dass historische Daten über einen Batch-Job
  verarbeitet werden. Neu anfallende Daten verarbeitet der sogenannte Speed- bzw.
  Realtime Layer und dieser füllt so die entstehende Lücke auf. Kombiniert werden
  die beiden Sichten auf die Daten vom \textit{Serving Layer}.

  \begin{figure}[h]
    \center
    \scalebox{.7}{\begin{tikzpicture}[>=stealth]
	\usetikzlibrary{shapes}
	\node (input)			at (3,3) [rounded rectangle, draw, minimum height=1cm,thick] {Input};
	\node (batch)			at (0,0) [cylinder, thick, shape border rotate=90, shape aspect=0.3, draw, minimum height=2.5cm, minimum width=3.5cm] {Batch-Layer};
	\node (serving)		at (0,-3) [rectangle, draw, minimum height=2cm, minimum width=3.5cm,thin] {Serving-Layer};
	\node (speed)			at (6,0) [rectangle, draw, minimum height=2cm, minimum width=3.5cm,thin] {Speed-Layer};
	\node (client)		at (3,-6) [rounded rectangle, draw, minimum height=1cm, minimum width=10cm,thick] {Client};

		\path (input)		edge[->,bend right,dashed,thick] (batch)
		(batch)					edge[->,thick] (serving)
		(serving)				edge[->,dashed,thick] (client)
		(input)					edge[->,bend left,dashed,thick] (speed)
		(speed)					edge[->,dashed,thick] (client);
\end{tikzpicture}
}
    \caption{Schematische Darstellung der Lambda-Architektur}
    \label{fig:lambdaarch}
  \end{figure}

  \subsection{Batch Layer}
  Der Batch-Layer geht davon aus, dass es ok ist, wenn es keine aktuelle Sicht
  auf die Daten gibt und die Berechnungen dadurch entsprechend einfach werden.
  Das heißt der Batch Job läuft, führt seine Berechnungen aus und schreibt das
  Ergebnis in eine Datenbank. Dabei werden schon in der Datenbank bestehende
  Daten vollständig ersetzt. Im Anschluss wird die Neuberechnung angestartet, die
  die in der Zwischenzeit angefallenen Daten mitbetrachtet. Ist diese
  abgeschlossen ersetzt das Ergebnis wieder das vorherige usw.

  \subsection{Speed / Realtime Layer}
  Der Speed-Layer füllt die Versorgungslücke, die entsteht, während die
  Berechnungen des Batch-Layers laufen. Das bedeutet der veränderbare Zusatand,
  der aus der Batch-Verarbeitung heraus gehalten wurde und die damit
  einhergehenden inkrementellen Algorithmen befinden sich jetzt im Speed-Layer
  und führen hier die Berechnungen in Echtzeit aus.

  Die Verschiebung der Komplexität in den Speed-Layer bringt einige Vorteile mit
  sich: Die Daten, die vom Speed-Layer berechnet werden, sind nur so lange
  gültig, wie der Batch-Layer für seine Berechnungen braucht. Das bedeutet ein
  Fehler bei der Datenanalyse kann den Betriebsfluss zwar stören - verschwindet
  aber sobald das Batch-System die Daten in der Datenbank unter Einbeziehung des
  neuen Wertes schreibt.

  An dieser Stelle ist das System auch Fehlern des Entwicklers über toleranter -
  es kann a) ein Fehler bei der Entwicklung der Verarabeitungslogik im
  Batch-Layer programmiert werden, oder b) bei der Entwicklung des Speed-Layers.

  Tritt der Fehler im Batch-Layer auf, so zieht sich der Fehler durch das gesamt
  System und verfälscht die Auswertungen. Da jede Neuberechnung allerdings auf
  sämtlichen historischen Daten basiert (Append-only) kann der Fehler vom
  Entwickler korrigiert werden und die Datenbasis ist nach der Ausbringung der
  Fehlerkorrektur und der nächsten Neuberechnung wieder korrekt. Zukünftige
  Berechnungen des Speed-Layers und Abfragen auf den Datenbestand nutzen die
  korrigiert Version, das heißt es gibt nur temporär eine verfälschte Datenbasis.

  Beim Auftreten eines Fehlers im Speed-Layer, so ist wie oben bereits
  angesprochen, nicht die gesamte Datenbasis vom Fehler beeinflusst, sondern nur
  der Zeitraum, seit die letzte Komplettberechnung des Batch-Layers abgeschlossen
  wurde. Davon ausgehend, dass die Berechnungen auf den Daten im Batch-Layer
  korrekt sind, sind somit die historischen Daten nicht vom Fehler beeinflusst.

  \subsection{Serving Layer}
  Die eigentliche Komplexität der Lambda-Architektur verbirgt sich hinter dem
  \textit{Serving Layer}. Dieser wird im ursprünglichen Artikel von Nathan Marz
  nicht beschrieben und wurde erstmals XXXX genannt. 

  Der Client des verteilten Systems soll im Optimalfall von der Aufteilung der
  Berechnungen in den Batch-Layer und den Speed-Layer nichts mitbekommen. Das
  heißt er muss eine übergreifende Sicht über beide zum Einsatz kommende
  Datenbanken bekommen und Abfragen müssen die Daten aus beiden Datenbanken
  zusammenfassen.

  Die Aufgabe des Serving-Layers ist es, dem Client diese Sicht zur Verfügung zu
  stellen und die Komplexität der Berechnungen im Hintergrund zu verbergen.

  \section{Verarbeitung von Streams}
  Das vorangegangene Kapitel zeigt, dass die Verarbeitung von Batch-Daten der
  Verarbeitung von Streams aufgrund der Komplexität vorzuziehen ist. Beide
  Verfahren können klar voneinander abgegrenzt werden: Bei der Batchverarbeitung
  ist die Größe der Daten bekannt, die Menge der Daten ist endlich und die
  Datenmenge kann „vollständig“ verarbeitet werden.

  Die Streamverarbeitung betrachtet im Gegensatz dazu die Verarbeitung von Daten,
  die erst während eines zeitlichen Verlaufs verfügbar werden. Statt die Daten,
  wie beim klassischen Data-Warehouse, erst zu indizieren und danach Abfragen auf
  den gespeicherten Daten auszuführen, werden die Daten im laufenden Betrieb mit
  statistischen und mathematischen Methoden analysiert.

  Der Betrachter der Ergebnisse erhält so in Echtzeit Rückmeldung über ein
  laufendes System und kann aus den ihm so zur Verfügung gestellten Informationen
  Handlungen ableiten.

  Zusätzlich existiert noch das sogenannte \textit{Micro-Batching}. Hier wird
  nicht jedes Event / jede Nachricht individuell betrachtet, sondern es werden
  mehrere Nachrichten gesammelt, um dann Berechnungen über diesen kleinen
  Datenbestand auszuführen. Die aufprojezierte Zeispanne über welche die Daten zu
  einem Batch zusammengefasst werden nennt sich \textit{Window}. Im folgenden
  sind zwei bekannte Windowing-Verfahren theoretisch vorgestellt, das Sliding
  Window wird im Praxisteil konkret eingesetzt.

  \subsection{Tumbling Window}
  Ein \textit{Tumbling Window} ist eine Zeitreihe von
  aufeinanderfolgenden und sich nicht überlappenden Intervallen fixer
  Größe.
  \begin{figure}[!h]
    \centering
    \begin{tikzpicture}
       %draw horizontal line   
      \draw[|->, -latex] (0,0) -- (7,0);
      \draw[-, dashed] (-1,0) -- (0,0);

       %draw numbers
      \foreach \x  in {0,...,7} {% 
        \draw (\x,0) node[below=7pt,anchor=north,xshift=0,rotate=0] {\x}; 
        \draw[] (\x,-0.1) -- (\x,0.1);
      }
      % draw braces
      \draw[decoration={brace,mirror,raise=0pt},decorate]
      (0,-1) -- node[below=6pt] {Window 1} (2,-1);
      \draw[decoration={brace,mirror,raise=0pt},decorate]
      (2,-1) -- node[below=6pt] {Window 2} (4,-1);
      \draw[decoration={brace,mirror,raise=0pt},decorate]
      (4,-1) -- node[below=6pt] {Window 3} (6,-1);
    \end{tikzpicture}
    \caption{Schematische Darstellung eines Tumbling Window}
  \end{figure}

  Interpretiert man die im Schaubild dargestellten Intervalle als
  Sekunden, so werden alle Ereignisse von Sekunde 0 bis Sekunde 2
  zusammengefasst, und im Anschluss als Ergebnis geliefert. Im
  Anschluss werden die Ereignisse von Sekunde 2 bis Sekunde 4
  zusammengefasst usw. Das Ergebnis ist immer das des letzten
  abgeschlossenen Intervalls. Jedes Ereignis wird nur einmal
  betrachtet. Das bedeutet, dass jedes Ergebnis maximal 2 Sekunden
  veraltet sein kann.

  \subsection{Sliding Window}
  Ein Sliding Window erzeugt nur ein Ergebnis, wenn in der
  betrachteten Zeit tatsächlich ein Ereignis aufgetreten ist. Die
  Ereignisse können zu mehr als einem Fenster gehören und das Fenster
  wird kontinuierlich um ein bestimmten Wert nach vorne geschoben.
  \begin{figure}[!h]
    \centering
    \begin{tikzpicture}
       %draw horizontal line   
      \draw[|->, -latex] (0,0) -- (7,0);
      \draw[-, dashed] (-1,0) -- (0,0);

       %draw numbers
      \foreach \x  in {0,...,7} {% 
        \draw (\x,0) node[below=7pt,anchor=north,xshift=0,rotate=0] {\x}; 
        \draw[] (\x,-0.1) -- (\x,0.1);
      }
      % draw braces
      \foreach \x  in {0,...,5} {% 
        \draw[decoration={brace,mirror,raise=0pt},decorate]
        (\x,-\x/3-1) -- node[below=6pt] {} (\x+2,-\x/3-1);
      }
      \draw (6,-3) node {Window $n$};
    \end{tikzpicture}
    \caption{Schematische Darstellung eines Sliding Window}
  \end{figure}

  Im Schaubild ist zu sehen, wie das Betrachtungsfenster mit der Zeit
  wandert, die einzelnen Fenster überschneiden sich. So kann ein
  Ereignis verschiedenen Fenstern zugeordnet werden. Das bedeutet, es
  wird immer eine aktuelle Sicht auf das Ergebnis präsentiert. Auch
  hier muss festgelegt werden, wie häufig das Ergebnis aktualisiert
  werden soll - es gibt also auch eine geringe Latenz.

% TODO Event / Nachricht definieren

% TODO Wo kommen diese Informationen her? Message Broker: AMQP/JMS-Style Broker
% vs. Log-Based-Message Broker

% TODO Wieso diese Frameworks? => Open Source, können problemlos verprobt
% werden

% TODO Avro / Serialisierung von Nachrichten

  \section{Apache Storm}

  Apache Storm bezeichnet eine von der Apache Software Foundation
  bereitgestellte kostenlose Open-Source Software. Dieses Framework
  erlaubt die Echtzeitverarbeitung von unbegrenzten Datenströmen und
  erreicht dabei einen Durchsatz von bis zu einer Million Tuple pro
  Sekunde pro Knoten des verteilten Systems. Das Framework wird auf
  der zugehörigen Website als skalierbar und fehlertolerant beworben.
  Zudem bietet es ein sogenanntes Guaranteed-Message-Processing, d.h.
  das Framework gewährleistet, dass Nachrichten auf jeden Fall
  verarbeitet werden und nicht verloren gehen.

  \subsection{Storm Komponenten}

  Der Einsatz des Storm Frameworks setzt die Kenntnis einiger
  Begrifflichkeiten voraus. Diese werden im Folgenden kurz eingeführt
  und erläutert.

  \paragraph{Tuple}
  Ein Tuple bezeichnet im Kontext des Frameworks eine Nachricht, die
  von diesem verarbeitet wird. Mathematisch ist ein Tupel allgemein
  definiert:
  \begin{quote}
    Ein Tupel besteht aus einer Liste endlich vieler, nicht
    notwendigerweise voneinander verschiedener Objekte.
  \end{quote}
  %\setlength{\abovedisplayskip}{-20pt}
  %\setlength{\abovedisplayshortskip}{-20pt}

  \begin{align*}
    (x_1, \ldots , x_n)
  \end{align*}

  \paragraph{Stream}
  Ein Stream fasst mehrere Tupel zu einem Datenstrom zusammen. Storm
  vearbeitet diese Streams.
  \begin{quote}
    [\ldots] einen kontinuierlichen Fluss von Datensätzen, dessen Ende
    meist nicht im Voraus abzusehen ist; die Datensätze werden
    fortlaufend verarbeitet, sobald jeweils ein neuer Datensatz
    eingetroffen ist.
  \end{quote}

  \begin{align*}
    (x_1, \ldots , x_n)
    (x_1, \ldots , x_n)
    (x_1, \ldots , x_n)
    (x_1, \ldots , x_n)
  \end{align*}
  \begin{figure}[!h]
    \centering
    \vspace*{-1cm}
    \begin{tikzpicture}
       %draw horizontal line   
      \draw[|->, -latex] (0,0) -- (7,0);
    \end{tikzpicture}
  \end{figure}

  \paragraph{Spout}
  Ein Spout, dt. „die Quelle“, erzeugt den zu verarbeitenden Stream.
  Quellen können verschiedene angebundene Systeme sein, zum
  Beispiel Message Queuing Systeme (Apache Kafka, Twitter Kestrel),
  Datenbanksysteme (Redis) oder Web-Ressourcen (Twitter).

  Storm stellt für die genannten Systeme bereits fertige Spouts zur
  Verfügung. Ein Spout ist sowohl für das Erzeugen neuer Nachrichten,
  als auch für das Verarbeiten der Rückmeldung, dass eine Nachricht
  erfolgreich verarbeitet wurde, zuständig. Analog dazu muss der Spout
  auch verloren gegangene Nachrichten neu senden. Dadurch wird das
  Guaranteed-Message-Processing möglich.

  % TODO Wie funktioniert das GMP?

  \paragraph{Bolt}

  Im Bolt findet die tatsächliche Verarbeitung statt.
  \begin{lstlisting}
void execute(Tuple input, BasicOutputCollector collector)
{
  /* do some magic stuff here, e.g:
  * - count words
  * - write to database
  * - combine data
  */
  ...
}
  \end{lstlisting}
  \paragraph{Topologie}
  \begin{figure}[!h]
    \center
    \scalebox{.7}{\begin{tikzpicture}[>=stealth]
	\usetikzlibrary{shapes}
	\node (spout0)			at (0,2.5) [rounded rectangle, draw, minimum height=1cm,thick] {Spout};
	\node (spout1)			at (0,-2) [rounded rectangle, draw, minimum height=1cm,thick] {Spout};

	\node (bolt0)			at (3,4) [rectangle, draw, minimum height=1cm,thick] {Bolt};
	\node (bolt1)			at (3,1) [rectangle, draw, minimum height=1cm,thick] {Bolt};
	\node (bolt2)			at (3,-2) [rectangle, draw, minimum height=1cm,thick] {Bolt};

	\node (bolt3)			at (6,2.5) [rectangle, draw, minimum height=1cm,thick] {Bolt};


	\path (spout0)		edge[->,thick] (bolt0)
		(spout0)				edge[->,thick] (bolt1)
		(spout0)				edge[->,thick] (bolt2)
		(spout1)				edge[->,thick] (bolt2)
		(bolt0)	  			edge[->,thick] (bolt3)
		(bolt1)	  			edge[->,thick] (bolt3);
\end{tikzpicture}
}
    \caption{Schematischer Aufbau einer Storm-Topologie}
    \label{fig:topology}
  \end{figure}

  \subsection{Guaranteed-Message-Processing}
  \paragraph{At-least-once-Semantik}
  \paragraph{Exactly-once-Semantik}

  \subsection{Apache Storm Trident} 

  \section{Echtzeitanalyse von Tastatureingaben}

% TODO Architekturschaubild
% TODO Zusammenarbeit jQuery, rest-Framework, Kafka, Storm

  \section{Zusammenfassung / Fazit}


  \end{document}
