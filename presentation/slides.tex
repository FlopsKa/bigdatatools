\documentclass{beamer}

\usepackage[utf8]{inputenc}
\usepackage[ngerman]{babel}
\usepackage[T1]{fontenc}
\usepackage{lmodern}
\usepackage{mathtools}
\usepackage{amssymb}
\usepackage{tikz}
\usepackage{graphicx}
\usepackage{tabularx}

% Presentation metadata
\title{Analyzing Big Data Streams}
\author{Florian Kalinke}
\date{18. Dezember 2017}

\begin{document}

\maketitle

\begin{frame}[t]{Übersicht}
  \tableofcontents
\end{frame}


\section{Stream-Verarbeitung}

\subsection{Abgrenzung zur Batch-Verarbeitung}

\begin{frame}[t]{Abgrenzung zur Batch-Verarbeitung}
  Ausschlaggebend für die Batch-Verarbeitung:
  \begin{itemize}
    \item Die Größe der Daten ist bekannt
    \item Die Menge der Daten ist endlich
    \item Die Datenmenge kann „vollständig“ verarbeitet werden
  \end{itemize}
  Die Streamverarbeitung betrachtet im Gegensatz dazu die Verarbeitung
  von Daten, die erst während eines \textit{zeitlichen Verlaufs} verfügbar
  werden.

  % Hier bekommt man schon langsam die Idee: Eigentlich will man beides haben
\end{frame}

\subsection{Einsatzgebiete}
\begin{frame}[t]{Einsatzgebiete}
\begin{itemize}
  \item Analyse von Sensordaten
  \item Echtzeitsysteme
  \item Fraud-Detection
\end{itemize}
\end{frame}

\subsection{Sliding / Tumbling Windows}
\begin{frame}[t]{Sliding / Tumbling Windows}
\begin{itemize}
  \item Sliding Window
    % TODO Bild einfügen
  \item Tumbing Window
    % TODO Bild einfügen
\end{itemize}
\end{frame}

\subsection{Lambda Architektur}
\begin{frame}[t]{Lambda Architektur}
  Der Name leitet sich aus dem Lambda-Kalkül als Grundlage der
  funktionalen Programmierung ab: Berechnungen im Batch-Layer werden
  nur auf unveränderlichen Daten ausgeführt.
\begin{figure}[h]
	\center
	\scalebox{.4}{\begin{tikzpicture}[>=stealth]
	\usetikzlibrary{shapes}
	\node (input)			at (3,3) [rounded rectangle, draw, minimum height=1cm,thick] {Input};
	\node (batch)			at (0,0) [cylinder, thick, shape border rotate=90, shape aspect=0.3, draw, minimum height=2.5cm, minimum width=3.5cm] {Batch-Layer};
	\node (serving)		at (0,-3) [rectangle, draw, minimum height=2cm, minimum width=3.5cm,thin] {Serving-Layer};
	\node (speed)			at (6,0) [rectangle, draw, minimum height=2cm, minimum width=3.5cm,thin] {Speed-Layer};
	\node (client)		at (3,-6) [rounded rectangle, draw, minimum height=1cm, minimum width=10cm,thick] {Client};

		\path (input)		edge[->,bend right,dashed,thick] (batch)
		(batch)					edge[->,thick] (serving)
		(serving)				edge[->,dashed,thick] (client)
		(input)					edge[->,bend left,dashed,thick] (speed)
		(speed)					edge[->,dashed,thick] (client);
\end{tikzpicture}
}
	\caption{Schematische Darstellung der Lambda-Architektur}
	\label{fig:lambdaarch}
\end{figure}
	% \includegraphics[width=\textwidth]{img/swhistfig.png}
\end{frame}

\section{Apache Storm Framework}
\begin{frame}[t]{Apache Storm Framework}
\end{frame}

\subsection{Komponenten}
\begin{frame}[t]{Komponenten}
\end{frame}

\subsection{Exactly-Once- / At-Least-Once-Semantik}
\begin{frame}[t]{Exactly-Once- / At-Least-Once-Semantik}
\end{frame}

\subsection{Storm Trident}
\begin{frame}[t]{Storm Trident}
\end{frame}

\section{Word Count Beispiel}
\begin{frame}[t]{Word Count Beispiel}
\end{frame}

\section{Quellen}
\begin{frame}[t]{Quellen}
\end{frame}

\end{document}


