\documentclass{beamer}

\usepackage[utf8]{inputenc}
\usepackage[ngerman]{babel}
\usepackage[T1]{fontenc}
\usepackage{lmodern}
\usepackage{mathtools}
\usepackage{amssymb}
\usepackage{tikz}
\usepackage{graphicx}
\usepackage{tabularx}
\usepackage{listings}
\usepackage{color}

\definecolor{dkgreen}{rgb}{0,0.6,0}
\definecolor{gray}{rgb}{0.5,0.5,0.5}
\definecolor{mauve}{rgb}{0.58,0,0.82}

\lstset{frame=tb,
  language=Java,
  aboveskip=3mm,
  belowskip=3mm,
  showstringspaces=false,
  columns=flexible,
  basicstyle={\small\ttfamily},
  numbers=none,
  numberstyle=\tiny\color{gray},
  keywordstyle=\color{blue},
  commentstyle=\color{dkgreen},
  stringstyle=\color{mauve},
  breaklines=true,
  breakatwhitespace=true,
  tabsize=3
}

% Presentation metadata
\title{Analyzing Big Data Streams}
\author{Florian Kalinke}
\date{18. Dezember 2017}

\begin{document}

\maketitle

\begin{frame}[t]{Übersicht}
  \tableofcontents
\end{frame}


\section{Stream-Verarbeitung}

\subsection{Einsatzgebiete}
\begin{frame}[t]{Einsatzgebiete}
\begin{itemize}
  \item Analyse von Sensordaten
  \item IoT Devices
  \item Fraud-Detection
  \item Complex Event Processing
  \item High Speed Trading Systeme
  \item Klickstream-Analyse

  \item Echtzeitsysteme
\end{itemize}
\end{frame}

\subsection{The Big Picture}
\begin{frame}[t]{The Big Picture}
  Bild von https://docs.microsoft.com/de-de/azure/stream-analytics/stream-analytics-introduction 
\end{frame}

\subsection{Microbatches - Sliding / Tumbling Windows}
\begin{frame}[t]{Sliding / Tumbling Windows}
\begin{itemize}
  \item Sliding Window
    % TODO Bild einfügen
  \item Tumbing Window
    % TODO Bild einfügen
\end{itemize}
\end{frame}

\subsection{Abgrenzung zur Batch-Verarbeitung}

\begin{frame}[t]{Abgrenzung zur Batch-Verarbeitung}
  Ausschlaggebend für die Batch-Verarbeitung:
  \begin{itemize}
    \item Die Größe der Daten ist bekannt
    \item Die Menge der Daten ist endlich
    \item Die Datenmenge kann „vollständig“ verarbeitet werden
  \end{itemize}
  Die Streamverarbeitung betrachtet im Gegensatz dazu die Verarbeitung
  von Daten, die erst während eines \textit{zeitlichen Verlaufs} verfügbar
  werden.

  % Hier bekommt man schon langsam die Idee: Eigentlich will man beides haben
\end{frame}


\subsection{Lambda Architektur}
\begin{frame}[t]{Lambda Architektur}
\begin{figure}[h]
	\center
	\scalebox{.4}{\begin{tikzpicture}[>=stealth]
	\usetikzlibrary{shapes}
	\node (input)			at (3,3) [rounded rectangle, draw, minimum height=1cm,thick] {Input};
	\node (batch)			at (0,0) [cylinder, thick, shape border rotate=90, shape aspect=0.3, draw, minimum height=2.5cm, minimum width=3.5cm] {Batch-Layer};
	\node (serving)		at (0,-3) [rectangle, draw, minimum height=2cm, minimum width=3.5cm,thin] {Serving-Layer};
	\node (speed)			at (6,0) [rectangle, draw, minimum height=2cm, minimum width=3.5cm,thin] {Speed-Layer};
	\node (client)		at (3,-6) [rounded rectangle, draw, minimum height=1cm, minimum width=10cm,thick] {Client};

		\path (input)		edge[->,bend right,dashed,thick] (batch)
		(batch)					edge[->,thick] (serving)
		(serving)				edge[->,dashed,thick] (client)
		(input)					edge[->,bend left,dashed,thick] (speed)
		(speed)					edge[->,dashed,thick] (client);
\end{tikzpicture}
}
	\caption{Schematische Darstellung der Lambda-Architektur}
	\label{fig:lambdaarch}
\end{figure}
  Der Name leitet sich aus dem Lambda-Kalkül als Grundlage der
  funktionalen Programmierung ab: Berechnungen im Batch-Layer werden
  nur auf unveränderlichen Daten ausgeführt.
	% \includegraphics[width=\textwidth]{img/swhistfig.png}
\end{frame}

\section{Apache Storm Framework}
\begin{frame}[t]{Apache Storm Framework}
  \begin{itemize}
    \item kostenlose open-source Software
    \item Echtzeitverarbeitung von „unbounded“ Streams
    \item bis zu 1M+ Tuple / Sekunde / Knoten
    \item skalierbar, fehlertolerant, guaranteed-message-processing
  \end{itemize}
\end{frame}

\subsection{Komponenten}
\begin{frame}[t]{Tuple}
  \begin{quote}
    Ein Tupel besteht aus einer Liste endlich vieler, nicht notwendigerweise voneinander verschiedener Objekte.
  \end{quote}

  \begin{align*}
    (x_1, \ldots , x_n)
  \end{align*}
\end{frame}

\begin{frame}[t]{Stream}
  \begin{quote}
    [\ldots] einen kontinuierlichen Fluss von Datensätzen, dessen Ende meist nicht im Voraus abzusehen ist; die Datensätze werden fortlaufend verarbeitet, sobald jeweils ein neuer Datensatz eingetroffen ist.
  \end{quote}

  \begin{align*}
    (x_1, \ldots , x_n)
    (x_1, \ldots , x_n)
    (x_1, \ldots , x_n)
    (x_1, \ldots , x_n)
  \end{align*}
\end{frame}

\begin{frame}[t]{Spout}
  Quelle der Streams - die Tupel werden von hier gesendet: 
  \begin{itemize}
    \item Kafka 
    \item Twitter 
    \item Kestrel 
    \item redis
    \item \ldots
  \end{itemize}
\end{frame}

\begin{frame}[fragile]{Bolt}
  Im Bolt findet die tatsächliche Verarbeitung statt.
  \begin{lstlisting}
void execute(Tuple input, BasicOutputCollector collector)
{
  /* do some magic stuff here, e.g:
  * - count words
  * - write to database
  * - combine data
  */
  ...
}

  \end{lstlisting}
\end{frame}

\begin{frame}[t]{Worker / Exekutoren / Tasks}
  Schaubild aus der Präsentation von der Homepage
\end{frame}


\subsection{Exactly-Once- / At-Least-Once-Semantik}
\begin{frame}[t]{Exactly-Once- / At-Least-Once-Semantik}
  Storm bietet „Guaranteed Message Processing“ $\Rightarrow$ allerdings mit
  At-Least-Once-Semantik
  \begin{itemize}
    \item Nachrichten werden auf jeden Fall verarbeitet
    \item im Fehlerfall wird die gleiche Nachricht nochmal gesendet
  \end{itemize}
  Werden die Tuple nicht einzeln, sondern als (Micro-)Batch verarbeitet, so bietet Storm Trident
  \begin{itemize}
    \item Non-Transactional
    \item Transactional
    \item Opaque Transactional
  \end{itemize}
  Verarbeitung. Damit ist eine Exactly-Once-Verarbeitung möglich.
\end{frame}

\subsection{Storm Trident}
\begin{frame}[t]{Storm Trident}
  Trident ist ein High-Level-Aufsatz auf Storm mit vorgefertigten Operationen:
  \begin{itemize}
    \item Merges / Joins
    \item Aggregationen (Count / Sum / Max)
    \item Gruppierungen
    \item Funktionen
    \item Filter
  \end{itemize}
\end{frame}

\subsection{Powered By Storm}
\begin{frame}[t]{Powered By Storm}
  \begin{itemize}
    \item Groupon
    \item Twitter
    \item Yahoo!
    \item Spotify
    \item Alibaba
    \item Yelp
    \item PARC
  \end{itemize}
\end{frame}

\section{Word Count Beispiel}
\begin{frame}[t]{Word Count Beispiel}
\end{frame}

\section{Quellen}
\begin{frame}[t]{Quellen}
\end{frame}

\end{document}


